\documentclass[chapter.computer_science.tex]{subfiles}
\begin{document}

\section{计算机网络}
\subsection{TCP连接管理}
\subsubsection{TCP的三次握手}
\begin{enumerate}
    \item 客户机向服务器发送一个连接请求报文段,此报文段不含应用层数据,SYN=1,客户端会选择一个随机的起始序seq=x。
    \item 服务端收到并同意后,发回确认,分配TCP缓存和变量,SYN=1, ACK=1,确认号为x+1,服务器随机产生起始需要seq=y,同样不包含应用层数据。
    \item 客户端收到后,向服务器发回确认,分配TCP缓存和变量,ACK=1,序号为x+1,确认号ack字段为y+1,该报文段可携带数据。
\end{enumerate}
\subsubsection{TCP的四次挥手}
\begin{enumerate}
    \item 客户机发送连接释放报文段,FIN=1, seq=u,TCP是全双工的,当发送FIN报文时,发送的那一端就不能再发送数据了,而接收的那一段此时仍可以发送数据。
    \item 服务端发送确认,确认号为u+1,此时TCP连接处于半关闭状态,服务端发送数据,客户端仍要接收。
    \item 若服务器已经没有向客户端发送的数据,那就通知TCP释放连接,此时发送FIN=1的连接释放报文段。
    \item 客户端收到连接释放报文段之后,必需发送确认,ACK=1, seq=u+1,此时TCP连接没有被释放掉,等待2MSL后客户端关闭连接。
\end{enumerate}

\end{document}
